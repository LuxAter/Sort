\documentclass{subfile}

\begin{document}
   \subsection{Bubble Sort}
   \label{sub:bubble_sort}
   
   \begin{align*}
     \text{Best-Case Preformance}\quad &O\left(n\right)\\
     \text{Average Preformance}\quad &O\left(n^2\right)\\
     \text{Worst-Case Preformance}\quad &O\left(n^2\right)\\
     \text{Worst-Case Space Complexity}\quad &O\left(1\right)
   \end{align*}

   \begin{multicols}{2}
     \subsubsection{Introduction}
     \label{ssub:introduction}
     
     Bubble sort is an algorithm that goes item by item, sorting pairs of two
     adjacent items at a time. This is where the name \textit{Bubble sort}
     originates from, due to the fact that the algorithm creates bubbles around
     two adjacent items at a time and conly considers the order of these two.
     This is one of simplest comprison sorting algorithms, as it can be
     inplemented in a few number o flines of code, and is simple to
     conceptualize. The algorithm runs through the list of items swaping pairs
     as needed, untill no more swaps were made on a single pass, which would
     mean that each item is in its position.

     The worst case complexity of bubble sort is $O(n^2)$, because each item
     can only be moved one positon in a pass, meaning that for each item
     that is out of position must be slowly moved into position. For even
     relativly small lists of data, this can take times which are far to large
     to be useful beyond a demonstration. There is one case where bubble sort
     can provide an advantage over other sorting algorithms. Because the entire
     process of the sotring, is checking if the items are in a sorted order,
     if the list is already in a sorted order, Bubble sort will make a single
     pass testing if everything is in a sorted order. After it finishes the
     single pass it will determin that the list is already sorted, and end.

     Bubble sort has two elements that cause sorting to be extreamly slowed.
     These are ``turtles'', and ``rabbits''.
     
     ``Rabbits'' are elements that are at the beginnging of the list and belong
     at the end of the list. Because bubble sort moves from the first to the
     last element on a repeating order, these ``rabbits'' will quickly move
     from the front to the end, because everytime they are compaired to a
     nother value, they are found to be greater than that value, and are
     swaped, until they reach their final position. They are called
     ``rabbits'' due to the fact that they will move to their final destination
     on the furst pass, unless a ``faster rabbit'' is found. Below is an
     example of a ``rabbit'', where $9$ is the ``rabbit''.

     \begin{align*}
       925580\\
       \textcolor{red}{92}5580\\
       2\textcolor{red}{95}580\\
       25\textcolor{red}{95}80\\
       255\textcolor{red}{98}0\\
       2558\textcolor{red}{90}\\
       255809\\
     \end{align*}
     
     ``Turtles'' are items that are near the end of the list, that belong at
     the begining. These are called ``turtles'' due to the fact that they cause
     the algorithm to run for much longer, and cannot move quickly to their
     final destination. The ``turtle'' can only move once per pass, as they are
     only moved for a single comparison, before the bubble moves on to the next
     adjacent pair. These ``turtles'' are what commonly cause the worst case
     preformance that is found in Bubble sort. As they can strictly only move
     one positon for each pass, they can at most require $n-1$ passes to move
     to their final position, where $n$ is the number of terms in the list.
     Below is an example of a ``turtle'', where $0$ is the ``turtle''. Each row
     is a new pass of the algorithm, where the bubble has passed through all
     the elemnts, similar to that demonstrated in the example for ``rabbits'',
     essentialy there are 5 steps between each step shown in the folowing
     example. 

     \begin{align*}
       925580\\
       1st \cdots 2558\textcolor{red}{90}\\
       2ed \cdots 255\textcolor{red}{80}9\\
       3ed \cdots 25\textcolor{red}{50}89\\
       4th \cdots 2\textcolor{red}{50}589\\
       5th \cdots \textcolor{red}{20}5589\\
       025589\\
     \end{align*}

     \subsubsection{Algorithm}
     \label{ssub:algorithm}
     \begin{algorithm}[H]
       \caption{Bubble Sort}
       \label{alg:bubble_sort}
       \begin{algorithmic}
         \Function{Bubble}{$a$}
           \State $length \gets $length($a$)
           \While{$length\neq0$}
             \State $newLength \gets 0$
             \For{$i=1$ to $length-1$}
               \If{$a[i-1]>a[i]$}
                 \State Swap($a[i-1]$,$a[i]$)
                 \State $newLength \gets i$
               \EndIf
             \EndFor
             \State $length\gets newLength$
           \EndWhile
           \State \Return $a$
         \EndFunction
       \end{algorithmic}
     \end{algorithm}

     This is a simple pseudocode algorithm for a bubble sort implementation.
     The algorithm begins by getting a value for the length of the array
     ($length$). Then the algorithm runs while this value is not equal to zero.
     Every time the algorithm begins a new pass through the array, is creates a
     new length value ($newLength$). Then the algorithm runs through the values
     of the array from the first value to the value at $length$. The
     $newLength$ value is set to the position of the last swap. This is due to
     the fact that beyond that position, there where no new swaps, so the data
     must already be sorted. This practice is to remove some additional time,
     as there is no reason for the algorithm to continue to check the already
     sorted data. The $length$ value is set to the value of $newLength$, then
     the process begins. As the value of $length$ becomes shorter, the
     algorithm is getting closer to compleation. When the value of $length$
     becomes $0$, this means that there were no more swaps in the list, and
     that must imply that all the values have been sorted correclty.

     \subsubsection{Analysis}
     \label{ssub:analysis}
     
     \begin{tikzpicture}
       \begin{axis}[scaled ticks=false,
         width=\linewidth,
         samples=100,
         xmin=1,
         xmax=100,
         ymin=0,
         ymax=1000,
         domain=1:100,
         restrict y to domain=0:1000,
         xlabel={$Elements$},
         ylabel={$Operations$},
         xticklabels={,,},
         yticklabels={,,},
         ]
           \addplot[color=red]{x^2};
           \addlegendentry{$n^2$}
           \addplot[color=yellow]{x^2};
           \addlegendentry{$n^2$}
           \addplot[color=green]{x};
           \addlegendentry{$n$}
           \addplot[color=blue]{1};
           \addlegendentry{$1$}
       \end{axis}
     \end{tikzpicture}

     This is a plot for the theoretical run times that Bubble sort should
     provide based on the number of elements provided. As it can be seen the
     best and average cases (which are the same) grow extreamly rappidly. The
     the rare best case preformance (which only will occure if a single pass is
     neccasary, e.g. the array is already sorted) is much lower. The memory
     usage is constant at $1$ as the sorting algorithm does not need to store
     or make copies of the data, simply to view and make swaps to the data. The
     exponential nature of the average case time complexity, means that this
     algorithm will rappidly grow to an unsuable length of time required to
     run. As it can clearly be seen in the implementation data below.

     \subsubsection{Conclusion}
     \label{ssub:conclusion}
     
     The bubble sort algorithm, is extreamly inefficient, and should only be
     used for demonstrational proupuses, or if the data is near the correct
     order. Because the algorithm is effective at checking the set of data, and
     can simply move a position one or two spots with only a maximum of two or
     three passes required. However, anything beyond this causes the algorithm
     to grow exponentialy in time complexity.

     Implementation examples for Bubble sort can be found in
     \ref{app:algorithms:bubble_sort}.
   \end{multicols}
\end{document}
